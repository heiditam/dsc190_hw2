% Options for packages loaded elsewhere
\PassOptionsToPackage{unicode}{hyperref}
\PassOptionsToPackage{hyphens}{url}
%
\documentclass[
]{article}
\usepackage{amsmath,amssymb}
\usepackage{iftex}
\ifPDFTeX
  \usepackage[T1]{fontenc}
  \usepackage[utf8]{inputenc}
  \usepackage{textcomp} % provide euro and other symbols
\else % if luatex or xetex
  \usepackage{unicode-math} % this also loads fontspec
  \defaultfontfeatures{Scale=MatchLowercase}
  \defaultfontfeatures[\rmfamily]{Ligatures=TeX,Scale=1}
\fi
\usepackage{lmodern}
\ifPDFTeX\else
  % xetex/luatex font selection
\fi
% Use upquote if available, for straight quotes in verbatim environments
\IfFileExists{upquote.sty}{\usepackage{upquote}}{}
\IfFileExists{microtype.sty}{% use microtype if available
  \usepackage[]{microtype}
  \UseMicrotypeSet[protrusion]{basicmath} % disable protrusion for tt fonts
}{}
\makeatletter
\@ifundefined{KOMAClassName}{% if non-KOMA class
  \IfFileExists{parskip.sty}{%
    \usepackage{parskip}
  }{% else
    \setlength{\parindent}{0pt}
    \setlength{\parskip}{6pt plus 2pt minus 1pt}}
}{% if KOMA class
  \KOMAoptions{parskip=half}}
\makeatother
\usepackage{xcolor}
\usepackage[margin=1in]{geometry}
\usepackage{color}
\usepackage{fancyvrb}
\newcommand{\VerbBar}{|}
\newcommand{\VERB}{\Verb[commandchars=\\\{\}]}
\DefineVerbatimEnvironment{Highlighting}{Verbatim}{commandchars=\\\{\}}
% Add ',fontsize=\small' for more characters per line
\usepackage{framed}
\definecolor{shadecolor}{RGB}{248,248,248}
\newenvironment{Shaded}{\begin{snugshade}}{\end{snugshade}}
\newcommand{\AlertTok}[1]{\textcolor[rgb]{0.94,0.16,0.16}{#1}}
\newcommand{\AnnotationTok}[1]{\textcolor[rgb]{0.56,0.35,0.01}{\textbf{\textit{#1}}}}
\newcommand{\AttributeTok}[1]{\textcolor[rgb]{0.13,0.29,0.53}{#1}}
\newcommand{\BaseNTok}[1]{\textcolor[rgb]{0.00,0.00,0.81}{#1}}
\newcommand{\BuiltInTok}[1]{#1}
\newcommand{\CharTok}[1]{\textcolor[rgb]{0.31,0.60,0.02}{#1}}
\newcommand{\CommentTok}[1]{\textcolor[rgb]{0.56,0.35,0.01}{\textit{#1}}}
\newcommand{\CommentVarTok}[1]{\textcolor[rgb]{0.56,0.35,0.01}{\textbf{\textit{#1}}}}
\newcommand{\ConstantTok}[1]{\textcolor[rgb]{0.56,0.35,0.01}{#1}}
\newcommand{\ControlFlowTok}[1]{\textcolor[rgb]{0.13,0.29,0.53}{\textbf{#1}}}
\newcommand{\DataTypeTok}[1]{\textcolor[rgb]{0.13,0.29,0.53}{#1}}
\newcommand{\DecValTok}[1]{\textcolor[rgb]{0.00,0.00,0.81}{#1}}
\newcommand{\DocumentationTok}[1]{\textcolor[rgb]{0.56,0.35,0.01}{\textbf{\textit{#1}}}}
\newcommand{\ErrorTok}[1]{\textcolor[rgb]{0.64,0.00,0.00}{\textbf{#1}}}
\newcommand{\ExtensionTok}[1]{#1}
\newcommand{\FloatTok}[1]{\textcolor[rgb]{0.00,0.00,0.81}{#1}}
\newcommand{\FunctionTok}[1]{\textcolor[rgb]{0.13,0.29,0.53}{\textbf{#1}}}
\newcommand{\ImportTok}[1]{#1}
\newcommand{\InformationTok}[1]{\textcolor[rgb]{0.56,0.35,0.01}{\textbf{\textit{#1}}}}
\newcommand{\KeywordTok}[1]{\textcolor[rgb]{0.13,0.29,0.53}{\textbf{#1}}}
\newcommand{\NormalTok}[1]{#1}
\newcommand{\OperatorTok}[1]{\textcolor[rgb]{0.81,0.36,0.00}{\textbf{#1}}}
\newcommand{\OtherTok}[1]{\textcolor[rgb]{0.56,0.35,0.01}{#1}}
\newcommand{\PreprocessorTok}[1]{\textcolor[rgb]{0.56,0.35,0.01}{\textit{#1}}}
\newcommand{\RegionMarkerTok}[1]{#1}
\newcommand{\SpecialCharTok}[1]{\textcolor[rgb]{0.81,0.36,0.00}{\textbf{#1}}}
\newcommand{\SpecialStringTok}[1]{\textcolor[rgb]{0.31,0.60,0.02}{#1}}
\newcommand{\StringTok}[1]{\textcolor[rgb]{0.31,0.60,0.02}{#1}}
\newcommand{\VariableTok}[1]{\textcolor[rgb]{0.00,0.00,0.00}{#1}}
\newcommand{\VerbatimStringTok}[1]{\textcolor[rgb]{0.31,0.60,0.02}{#1}}
\newcommand{\WarningTok}[1]{\textcolor[rgb]{0.56,0.35,0.01}{\textbf{\textit{#1}}}}
\usepackage{graphicx}
\makeatletter
\def\maxwidth{\ifdim\Gin@nat@width>\linewidth\linewidth\else\Gin@nat@width\fi}
\def\maxheight{\ifdim\Gin@nat@height>\textheight\textheight\else\Gin@nat@height\fi}
\makeatother
% Scale images if necessary, so that they will not overflow the page
% margins by default, and it is still possible to overwrite the defaults
% using explicit options in \includegraphics[width, height, ...]{}
\setkeys{Gin}{width=\maxwidth,height=\maxheight,keepaspectratio}
% Set default figure placement to htbp
\makeatletter
\def\fps@figure{htbp}
\makeatother
\setlength{\emergencystretch}{3em} % prevent overfull lines
\providecommand{\tightlist}{%
  \setlength{\itemsep}{0pt}\setlength{\parskip}{0pt}}
\setcounter{secnumdepth}{-\maxdimen} % remove section numbering
\ifLuaTeX
  \usepackage{selnolig}  % disable illegal ligatures
\fi
\IfFileExists{bookmark.sty}{\usepackage{bookmark}}{\usepackage{hyperref}}
\IfFileExists{xurl.sty}{\usepackage{xurl}}{} % add URL line breaks if available
\urlstyle{same}
\hypersetup{
  pdftitle={Video Game Preferences Among UC Berkeley Stats Students in Fall 1994},
  pdfauthor={Heidi Tam and Paige Pagaduan},
  hidelinks,
  pdfcreator={LaTeX via pandoc}}

\title{Video Game Preferences Among UC Berkeley Stats Students in Fall
1994}
\author{Heidi Tam and Paige Pagaduan}
\date{2024-10-19}

\begin{document}
\maketitle

\hypertarget{contribution-statement}{%
\section{0. Contribution Statement}\label{contribution-statement}}

\pagebreak

\pagenumbering{arabic}

\hypertarget{introduction}{%
\section{Introduction}\label{introduction}}

\hypertarget{data}{%
\subsubsection{Data}\label{data}}

The data from videodata.txt and videoMultiple.txt were collected as part
of a survey for students at UC Berkeley enrolled in a particular
statistics course that had about 3000-4000 students. Students from
Statistics 2, Section 1, in Fall 1994, were invited to participate in
the survey if they partook in the second exam of the course. Within this
section, 314 students were eligible to participate and 95 students were
randomly selected through a random number generator. The data from
videodata.txt was the first part of the survey and asked about
background information of the survey participant. The data was numerical
and discrete (e.g., Time, the number of hours played in the week prior
to the survey as an integer) or categorical but encoded as a numerical
value (e.g., Like to play, rated on a scale of 1 to 5). The data from
videoMultiple.txt was the second part of the survey and covered whether
the student likes or dislikes playing video games. In these questions,
more than one response could be given. The data in this file was
recorded as binary values indicating whether the student selected that
option. Some columns also contain string values if the student provided
a reason that was not given as to why they dislike video games.

\pagebreak

\hypertarget{basic-analysis}{%
\section{Basic Analysis}\label{basic-analysis}}

\hypertarget{question-01}{%
\subsection{Question 01}\label{question-01}}

\hypertarget{methods}{%
\subsubsection{Methods}\label{methods}}

First, we loaded the data in through R.

To determine a point estimate of the fraction of students we played a
video game in the week prior to the survey, we need the total number of
students. The video\_data table has 91 observations and 15 variables.
Since we have no way of proving that each student only filled out the
form once, we will assume so for this assignment.

\begin{Shaded}
\begin{Highlighting}[]
\NormalTok{n }\OtherTok{\textless{}{-}} \FunctionTok{nrow}\NormalTok{(video\_data)}
\CommentTok{\# POINT ESTIMATE:}
\CommentTok{\# We can count the number of students who played a video game in the week prior to the survey by counting the number of students who played a video game for more than zero hours in the week prior. }
\NormalTok{played\_count }\OtherTok{\textless{}{-}} \FunctionTok{sum}\NormalTok{(video\_data}\SpecialCharTok{$}\NormalTok{time }\SpecialCharTok{\textgreater{}} \DecValTok{0}\NormalTok{) }
\NormalTok{point\_estimate\_fraction }\OtherTok{\textless{}{-}}\NormalTok{ played\_count }\SpecialCharTok{/}\NormalTok{ n}

\CommentTok{\# INTERVAL ESTIMATE:}
\CommentTok{\# We will construct a confidence interval that contains a range of values that likely contain the population parameter, the proportion of all students who answered the survey who played a video game in the week prior to the survey. }
\CommentTok{\# For a 95\% confidence interval, we will use z = 1.96.}
\NormalTok{z }\OtherTok{\textless{}{-}} \FloatTok{1.96}
\NormalTok{lower\_interval\_estimate\_fraction }\OtherTok{\textless{}{-}}\NormalTok{ point\_estimate\_fraction }\SpecialCharTok{{-}}\NormalTok{ z }\SpecialCharTok{*} \FunctionTok{sqrt}\NormalTok{(point\_estimate\_fraction }\SpecialCharTok{*}\NormalTok{ (}\DecValTok{1} \SpecialCharTok{{-}}\NormalTok{ point\_estimate\_fraction) }\SpecialCharTok{/}\NormalTok{ n)}
\NormalTok{upper\_interval\_estimate\_fraction }\OtherTok{\textless{}{-}}\NormalTok{ point\_estimate\_fraction }\SpecialCharTok{+}\NormalTok{ z }\SpecialCharTok{*} \FunctionTok{sqrt}\NormalTok{(point\_estimate\_fraction }\SpecialCharTok{*}\NormalTok{ (}\DecValTok{1} \SpecialCharTok{{-}}\NormalTok{ point\_estimate\_fraction) }\SpecialCharTok{/}\NormalTok{ n)}
\end{Highlighting}
\end{Shaded}

We constructed the confidence interval using the formula p-hat = z +/-
sqrt((p-hat)*(1 - p-hat) / n), where p-hat is the sample proportion of
students who played a video game in the past week and n is the sample
size. Z is the Z-score for the confidence level we chose (95\%).

\end{document}
